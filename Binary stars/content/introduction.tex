\section{Theorie}
\label{sec:Theorie}

%\cite{sample}
Overcontact binary stars are systems where both components share an envelope and 
are in physical contact with each other. This leads to luminosity and mass exchange 
between the stars. W Ursae Majoris (W UMa) binary systems are well known overcontact 
systems that have F, G and K-type stars with common envelopes. It is important to 
understand their internal structure and evolution to be able to model their observed 
properties (Terrell et al. 2012). SW Lacertae is a short-period contact binary star 
system (P 0.d32, Vmax=8.m91). The variability of the system was first discovered in 
1918 by Miss Ashall (Leavitt 1918), evaluating photo plates. The first photoelectric 
UBV light curves of the system were collected by Brownlee (1956). They highlighted the 
light curve asymmetries from cycle to cycle and these asymmetries were confirmed and 
assigned to the existence of cool spot regions by multiple authors (Albayrak et al. 
2004, Alton \& Terrell 2006, and references therein). The distance from SW Lacertae to 
earth is 70.773 parsec and its mean magnitude is 8.51. As it is common for contact 
binaries, the two minima of one period are from different magnitudes. Due to it’s 
high visual magnitude changes, $\delta mag = 0.8$, the system is well studied. It has a 
period of 0.32 day and a orbital inclination of 80°. Seasonal light curve asymmetries,
 which are usual for over-contact binaries, were observed when Brownlee (1956) used the
  photoelectric effect for his observations. This effect is called O’Connel effect and 
  it is connected to cool star spots (Liu et al. 2003). 
In the following, the phase of the binary star system is studied.    