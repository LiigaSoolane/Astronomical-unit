\section{Data Reduction}
\label{sec:data}

The data collected was imported into IRAF to correct the object images using our Bias, 
Dark and Flat images. The images that were affected by the camera shutter malfunctioning 
had to be discarded. This caused the stars in our images to streak and added uneven exposure 
to our image frame. After discarding our bad files, the remaining observations were run 
through the CCDPROC procedure in IRAF to reduce the object data. The first step was to 
set our instrument to the “blt” using “setinstrument” in IRAF. Then run zerocombine, 
darkcomine, flatcombine and ccdproc (found in the packages NOAO, IMRED, and CCDRED). 
Zerocombine selects the bias images from our dataset and combines them into a master 
bias image. Darkcombine selects the dark images to create a master dark and subtracts 
the master bias, then scales them by exposure time and combines them. Flatcombine selects 
flat-field images then bias and dark subtracts them, scales to the modes and combines 
them in each filter to create master flats in each filter. Next, we used the task setjd 
(found in the ASTUTIL package) to calculate and add the Julian Date keywords to our dataset. 
Within the parameters for this task we set the observatory to “aro”, the date keyword, the 
“ut” time keyword, exposure time keyword, right ascension (in hours) keyword, and the declination 
(in degrees) keyword. Additionally, we set the “utdate” and “uttime” parameters to yes 
because our observation date and time is in UT. \\

\noindent The field is in two different orientations caused by flipping the telescope at the meridian. 
The meridian flip began at $20211105_ 1055$.fit. The object was labeled as $1$ and two comparison 
stars and $2$ and $3$. One image was chosen before the meridian flip and one image after to record
the x and y coordinates and the Full Width at Half Max (FWHM) value for the object and 
comparison stars. For the image East of the Meridian our x and y coordinate for the 
object was $(489.89, 800.88)$, for comparison star $1$ was $(235.13, 460.29)$ and for comparison 
star $2$ was $(762.15, 365.42)$. The FWHM value for the object was $4.80$, for comparison star 
$1$ was $5.03$ and for comparison star $2$ was $5.00$. For the image West of the Meridian our x and 
y coordinate for the object was $(425.08, 95.35)$, for comparison star 1 was $(682.16, 434.13)$ 
and for comparison star 2 was $(156.04, 532.72)$. The FWHM value for the object was 5.26, for 
comparison star $1$ was $4.95$ and for comparison star 2 was $4.72$. The average FWHM value was $4.92$. 
To confirm the FWHM value was approximately uniform throughout the observation window we 
checked several random images. To confirm the autoguider was functioning properly we used 
imexamine to verify the coordinates of the three stars matched within 5 pixels. \\

\noindent As the final step, photometry was performed on the object and comparison stars. 
The aperture was set to the FWHM average value and the annulus to three times of 
FWHM average value. The IRAF command “phot” was used non-interactively with east 
and west lists that contained the coordinates of the object and comparison stars 
before and after the meridian flip. Finally, the IRAF command “txdump” was used to 
extract the photometry data from the mag-files to analyze the final data. 
