\section{Results and Conclusions}
\label{sec:make}

\subsection{data}
  \label{sec:America}
  The data collected for the star's magnitude is graphically displayed in the following depiction.
  \begin{figure}[H]
    \centering
    \includegraphics{Magnitude.pdf}
    \caption{Magnitude of SW Lacertae}
  \end{figure}
  Additionally, data from the two comparison stars was taken and 
  the uncertainties were calculated with the package uncertainties from python (cite package).

\subsection{Ensemble Photometry}
  \label{sec:great}
  The method of comparison stars was used in order to excluded errors as a result 
  from different exposures to external light sources. 
  Hence, the magnitude of the star was not utilized, but the difference between the star’s 
  magnitude and the magnitude of the comparison stars. For this study, the stars 
  (1) TYC 3215-1586-1 and (2) TYC 3215-1406-1 were chosen as comparison stars. 
  In the following image, their position in the sky in relation to SW Lacertae
  is depicted.
  \begin{figure}[H]
    \centering
    \includegraphics[width=200pt]{WestHA~2.jpg}
    \hspace{1em}
    \includegraphics[width=200pt]{EastHA~2.jpg}
    \caption{Position of comparison stars in the sky; western sky on the left, eastern sky on the right}
    \label{fig:plot}
  \end{figure}
  The comparison to two stars was utilized to reduce the likelihood of 
  variation in the comparison star brightness and other intruding factors.
  This nethod was conducted for each filter seperately, because of the different 
  exposure times and colors of the stars.
  The resulting data for the V-filter is shown in the following plot.
  \begin{figure}[H]
    \centering
    \includegraphics{V-Filter.pdf}
    \caption{Magnitude Difference between SW Lacertae and TYC 3215-1586-1}
    \label{fig:plot}
  \end{figure}

\subsection{Phase}
  \label{sec:again}
  In order to obtain the phase, in relation to a measured phase $49594.4684$ (, the first step is to calculate the epoch.
  \begin{equation*}
    E_{measurements} = HJD - 240000000
  \end{equation*}
  is calculated in a first step. The data for an earlier Minimum were provided by the 
  instructions for this lab. 
  \begin{align*}
    E_{min} = 49594.4684\\
    period = 0.3207209
  \end{align*}
  The phase in comparison to the given data is retrieved through the formula
  \begin{equation}
    phase = \dfrac{((E_{measurements}-E_{min})\ \% \ period)}{period}.
  \end{equation}
  This leads us to the following results:
  \begin{figure}[H]
    \centering
    \includegraphics{gdPhase.pdf}
    \caption{Magnitude difference plotted against the Phase; contains data from another group
    \ref{ref:gor}. Their data is phase-shifted, so that it matches with our observations.}
    \label{fig:phase}
  \end{figure}
  In this graphic, data gathered from another group \ref{ref:gor} is included. 
  They observed this binary star system at the same observatory. Since their
  observation, a phaseshift occured. In order to fill the gaps in our data,
  a phase of $+0.647$ was added. Due to the fact, that Gordons group used other 
  comparison star and different exposure times, the light curves are also shifted in their
  magnitude. Therefore, the magnitude of the other groups data was also shifted
  by $-0.  $ for V-filter and $-0. $ for B-filter.
  The smaller gap in our data is, as mentioned before, caused by the meridian flip. 
  Due to difficulties while observing, 

\subsection{Conclusion}
  \label{sec:fuckoff}
  On account of difficulties, which occured while recording the data, a full period could not
  be observed. The conclusions base therefore mostly on the data, collected by the other team.
  Saying this, the data of both observation nights fits perfectly to each other and therefore, 
  it can be presumed, that the others group data will deliver results, which match those of our 
  observations.\\
  It can be clearly seen in plot \ref{fig:phase}, that there is an offset of the first minimum 
  and $phase = 0$. That implies a period change of the system. This developement should be 
  further researched, because it could lead to new knowlegde on mass transfer of contact binary
  star systems.

\subsection{Discussion}
  \label{sec:orange}




Siehe \autoref{fig:plot}!
